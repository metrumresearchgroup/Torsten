% Created 2022-05-18 Wed 13:44
% Intended LaTeX compiler: pdflatex
\documentclass[12pt, reqno, oneside]{amsbook}
              \newcommand\subtitle[1]{\newcommand\mrgsubtitle{#1}}
\newcommand\mrgproject{}
\newcommand\mrgtitle{Torsten: A Pharmacokinetic/Pharmacodynamic Model Library for Stan}
\newcommand\mrgsubtitle{\large{Developers Guide} \linebreak (Torsten Version 0.91.0, Stan version 2.33.1)}
\include{mrgtitlepage}
\usepackage{imakeidx}
\makeindex
\usepackage[letterpaper, width=6.5in, height=9in]{geometry}
\usepackage{graphicx}
\usepackage{pdfpages}
\usepackage{amssymb}
\usepackage{epstopdf}
\usepackage{xcolor}
\definecolor{MRGGreen}{rgb}{0, 0.350, 0.200}
\usepackage[colorlinks=true, citecolor=MRGGreen, urlcolor=MRGGreen, linkcolor=MRGGreen]{hyperref}
\usepackage{bold-extra}
\usepackage{courier}
\usepackage{listings}
\usepackage{siunitx}
\usepackage{booktabs}
\usepackage[framemethod=TikZ, skipabove=10pt, skipbelow=10pt, backgroundcolor=black!3, roundcorner=4pt, linewidth=1pt]{mdframed}
\BeforeBeginEnvironment{minted}{\begin{mdframed}}
\AfterEndEnvironment{minted}{\end{mdframed}}
\usepackage{subcaption}
\renewcommand{\chaptername}{}
\numberwithin{equation}{chapter}
\numberwithin{figure}{chapter}
\numberwithin{table}{chapter}
\usepackage[section]{placeins}
\renewcommand{\thesection}{\thechapter.\arabic{section}}
\theoremstyle{remark}
\newtheorem{example}{Example}
\newtheorem{remark}{Remark}


\usepackage[utf8]{inputenc}
\usepackage[T1]{fontenc}
\usepackage{graphicx}
\usepackage{grffile}
\usepackage{longtable}
\usepackage{wrapfig}
\usepackage{rotating}
\usepackage[normalem]{ulem}
\usepackage{amsmath}
\usepackage{textcomp}
\usepackage{amssymb}
\usepackage{capt-of}
\usepackage{hyperref}
\usepackage[newfloat]{minted}
\usepackage{caption}
\setcounter{secnumdepth}{3}
\author{Yi Zhang}
\date{\today}
\title{Torsten: A Pharmacokinetic/Pharmacodynamic Model Library for Stan\\\medskip
\large Developers guide \\  (Torsten Version 0.90.0, Stan version 2.29.2)}
\hypersetup{
 pdfauthor={Yi Zhang},
 pdftitle={Torsten: A Pharmacokinetic/Pharmacodynamic Model Library for Stan},
 pdfkeywords={},
 pdfsubject={},
 pdfcreator={Emacs 27.2 (Org mode 9.4.4)},
 pdflang={English}}
\begin{document}

\maketitle
\tableofcontents


\chapter*{Development team}
\label{sec:org9bb72d8}
\begin{itemize}
\item \href{mailto:billg@metrumrg.com}{William R. Gillespie} , \href{https://www.metrumrg.com/}{Metrum Research Group}
\item \href{mailto:yz@yizh.org}{Yi Zhang} , \href{https://www.c-path.org/}{Critical Path Institute}
\item \href{mailto:cmargossian@flatironinstitute.org}{Charles Margossian} , Flatiron Institute
\end{itemize}
\chapter*{Acknowledgements}
\label{sec:org4fd8ba7}
\section*{Institutions}
\label{sec:orgc7e58d4}
We thank Metrum Research Group, Columbia University, and AstraZeneca.
\section*{Funding}
\label{sec:org209e741}
This work was funded in part by the following organizations:
\subsection*{Office of Naval Research (ONR) contract N00014-16-P-2039}
\label{sec:orgee502e1}
provided as part of the Small Business Technology Transfer (STTR)
program. The content of the information presented in this document
does not necessarily reflect the position or policy of the
Government and no official endorsement should be inferred.
\subsection*{Bill \& Melinda Gates Foundation.}
\label{sec:orgb66b095}
\section*{Individuals}
\label{sec:org431c2f6}
We thank the Stan Development Team for giving us guidance on how to
create new Stan functions and adding features to Stan's core language
that facilitate building ODE-based models.
\chapter{Introduction}
\label{sec:orgca864c7}
Following Stan's code structure, Torsten forks four Stan repositories:
\begin{itemize}
\item The automatic differentiation and mathematical infrastructure \href{https://github.com/metrumresearchgroup/math}{\texttt{Stan/Math}}.
\item The inference engine and statistical model infrastructure \href{https://github.com/metrumresearchgroup/stan}{\texttt{Stan}}.
\item Platform-independent command line interface \href{https://github.com/metrumresearchgroup/cmdstan}{\texttt{CmdStan}}.
\item Stan language to C++ transpiler \href{https://github.com/metrumresearchgroup/stanc3}{\texttt{stanc3}}.
\end{itemize}

Same as Stan, the forked repos are in a submodule hierarchy
\begin{minted}[breaklines=true,fontsize=\footnotesize,breakanywhere=true]{bash}
Torsten/cmdstan/stan/lib/stan_math
\end{minted}

Additionally Torsten's own mathematical functions are in a separate
repository. It is a submodule of \texttt{Stan/Math} that
contains all of Torsten functions

\begin{minted}[breaklines=true,fontsize=\footnotesize,breakanywhere=true]{bash}
Torsten/cmdstan/stan/lib/stan_math/stan/math/torsten
\end{minted}
as well as tests.
\begin{minted}[breaklines=true,fontsize=\footnotesize,breakanywhere=true]{bash}
Torsten/cmdstan/stan/lib/stan_math/stan/math/torsten/test
\end{minted}

The rest of this document describes the basic layout of the components
in Torsten implementation. The details can be found in code comments.

\chapter{\texttt{torsten\_math}}
\label{sec:org70ecd1c}
Most Torsten development occurs in the \texttt{torsten\_math} repo. So far
most functions there are for solving PKPD ODE systems. To
understand the design, let us take a look at the user-facing \texttt{pmx\_solve\_rk45} function as an example.
\begin{minted}[breaklines=true,fontsize=\footnotesize,breakanywhere=true]{cpp}
  template <typename F, typename... Ts>
  auto pmx_solve_rk45(const F& f, const int nCmt, Ts... args) {
    using scheme_t = torsten::dsolve::odeint_scheme_rk45;
    return
    PMXSolveODE<dsolve::PMXOdeIntegrator<dsolve::PMXVariadicOdeSystem, dsolve::PMXOdeintIntegrator<scheme_t>>>::solve(f, nCmt, args...);
}
\end{minted}
The argument \texttt{f} is the ODE right-hand-side specification, \texttt{nCmt} is
the number of the states which equals to the size of \texttt{f}'s
output. Parameter pack \texttt{args} are for NMTRAN-compatible event
specification as well as ODE solver controls. The function is
just a wrapper of the line
\begin{minted}[breaklines=true,fontsize=\footnotesize,breakanywhere=true]{cpp}
PMXSolveODE<dsolve::PMXOdeIntegrator<dsolve::PMXVariadicOdeSystem,
dsolve::PMXOdeintIntegrator<scheme_t>>>::solve(f, nCmt, args...);
\end{minted}

\texttt{PMXOdeIntegrator} class describes the ODE integrator. It has two
template arguments. The first is \texttt{PMXVariadicOdeSystem} that will be
constructed using \texttt{f} and provides the ODE information (RHS, Jacobian,
etc). The second is \texttt{PMXOdeintIntegrator} that indicates we will be
using an integrator from \texttt{Boost::odeint} library. The specific
integrator is \texttt{torsten::dsolve::odeint\_scheme\_rk45}. An alternative
is
\begin{minted}[breaklines=true,fontsize=\footnotesize,breakanywhere=true]{cpp}
torsten::dsolve::odeint_scheme_ckrk
\end{minted}
used in \texttt{pmx\_solve\_ckrk}.

Here \texttt{PMXVariadicOdeSystem} is used to differentiate from the old ways
of using fixed parameters in ODE solvers, which is still used in unit tests.
The \texttt{PMXOdeintIntegrator} parameter is used to set apartment from
\texttt{CVODES} ODE solvers that \texttt{pmx\_solve\_bdf} and \texttt{pmx\_solve\_adams} are
based on.

\section{PMX solver}
\label{sec:org9dfe6ab}
The \texttt{PMXSolveODE} class uses the type of ODE integrator and ODE
system as arguments. It is the basis of all user-facing numerical
ODE functions. It provides a \texttt{static} function \texttt{solve} that uses
event schedule arguments and ODE controls to
\begin{itemize}
\item construct events from NMTRAN arguments,
\item construct requested ODE integrator,
\item construct event solvers according to the events and the type of integrator,
\item solve events chronologically.
\end{itemize}

A similar class \texttt{PMXSolveCPT} does the same but for compartment
models that employ close-form solutions.

\section{PMX ODE integrators}
\label{sec:org70bc868}
The purpose of \texttt{PMXOdeIntegrator} class is to separate the ODE solver,
the control parameters of numerical solutions, the ODE system, and the
user-facing functions. A \texttt{PMXOdeIntegrator} object is constructed
based on the type of ODE system (variadic or not), the type of solver
to be used (\texttt{Boost::odeint} vs \texttt{CVODES}), and the controls (tolerance
and maximum number of steps).

An \texttt{PMXOdeIntegrator} object is constructed inside
\texttt{PMXSolveODE::solve} for numerically solving ODEs.

\section{ODE integrators}
\label{sec:orga6a7e17}
\texttt{PMXOdeIntegrator} class delegates to specific types of ODE
integrators from \texttt{Boost::odeint} and \texttt{CVODES} libraries for actual numerical
solution. Thus it has a type argument that specifies the integrator
to be used. All the supported numerical solvers can be found in
\texttt{torsten\_math/dsolve}. Torsten uses its own implementation based on
library APIs instead of directly build upon Stan's ODE integration
functions.

\section{PMX models}
\label{sec:orgea0adf9}
In \texttt{PMXSolveCPT} and \texttt{PMXSolveODE} class, PMX models are constructed
using NMTRAN inputs. For example, the one-compartment model class has
the following construct.
\begin{minted}[breaklines=true,fontsize=\footnotesize,breakanywhere=true]{cpp}
template<typename T_par>
class PMXOneCptModel {
  const T_par &CL_;
  const T_par &V2_;
  const T_par &ka_;
  const T_par k10_;
  const std::vector<T_par> alpha_;
  const std::vector<T_par> par_;

public:
  static constexpr int Ncmt = 2;
  static constexpr int Npar = 3;
  static constexpr PMXOneCptODE f_ = PMXOneCptODE();

//...
};
\end{minted}
One can see that it stores model parameters clearance, volume of
distribution, absorption coefficient, which are used to construct
parameters of close-form solutions. The class also contains static
components such as number of compartments, number of parameters, and
the RHS of the one-compartment ODE system.

Here is a list of Torsten models ("/" indicate the model consisting
two coupled components)

\begin{table}[htbp]
\caption{Models supported in Torsten}
\centering
\begin{tabular}{lll}
Model & Class & File\\
\hline
One-cpt PK & \texttt{PMXOneCptModel} & \texttt{pmx\_onecpt\_model.hpp}\\
Two-cpt PK & \texttt{PMXTwoCptModel} & \texttt{pmx\_twocpt\_model.hpp}\\
Linear ODE & \texttt{PMXLinODEModel} & \texttt{pmx\_linode\_model.hpp}\\
One-cpt/effective-cpt & \texttt{PMXOneCptEffCptModel} & \texttt{pmx\_onecpt\_effcpt\_model.hpp}\\
Two-cpt/effective-cpt & \texttt{PMXTwoCptEffCptModel} & \texttt{pmx\_twocpt\_effcpt\_model.hpp}\\
General ODE & \texttt{PKODEModel} & \texttt{pmx\_ode\_model.hpp}\\
Close-form PK/general ODE & \texttt{PKCoupledModel} & \texttt{pmx\_coupled\_model.hpp}\\
\hline
\end{tabular}
\end{table}

Each model has an overloaded member function \texttt{solve} that solves a given
event. The overloading is for different signatures in transient and
steady-state solutions.

\section{Event management}
\label{sec:org45a94d9}
The \texttt{PMXSolveODE::solve} (\texttt{PMXSolveCPT::solve} is similar) function
looks like this.

\begin{minted}[breaklines=true,fontsize=\footnotesize,breakanywhere=true]{cpp}
template <typename T0, typename T1, typename T2, typename T3, typename T4,
          typename T5, typename T6, typename F>
static stan::matrix_return_t<T0, T1, T2, T3, T4, T5, T6>
solve(
// args...
     ) {
// ...

  using ER = NONMENEventsRecord<T0, T1, T2, T3>;
  using EM = EventsManager<ER, NonEventParameters<T0, T4, std::vector, std::tuple<T5, T6> >>;
  const ER events_rec(nCmt, time, amt, rate, ii, evid, cmt, addl, ss);

  Matrix<typename EM::T_scalar, -1, -1> pred(EM::nCmt(events_rec), events_rec.num_event_times());

  using model_type = torsten::PKODEModel<typename EM::T_par, F>;

  integrator_type integrator(rel_tol, abs_tol, max_num_steps, as_rel_tol, as_abs_tol, as_max_num_steps, msgs);
  EventSolver<model_type, EM> pr;

  pr.pred(0, events_rec, pred, integrator, pMatrix, biovar, tlag, nCmt, f);
  return pred;
}
\end{minted}
One can see that first a \texttt{NONMENEventsRecord} object is created using
the NMTRAN arguments (template parameter type T1-T6 are for these
arguments), then \texttt{EventsManager} and \texttt{EventSolver} objects are
constructed. The purpose of \texttt{EventsManager} is to create and sort
events chronologically, based on NMTRAN input. The purpose of
\texttt{EventSolver} is to solve the events using specified ODE solver.

\chapter{\texttt{stan/math}}
\label{sec:org7dc0359}
The \texttt{stan/math} repo serves as a pass-through fork for
\texttt{torsten\_math}. It is almost identical to Stan's upstream repo, with the
only noticeable difference in the \texttt{math/stan/math.hpp}, in which the
\texttt{torsten} namespace is added:

\begin{minted}[breaklines=true,fontsize=\footnotesize,breakanywhere=true]{cpp}
#ifndef STAN_MATH_HPP
#define STAN_MATH_HPP

/**
 * \defgroup prob_dists Probability Distributions
 */

/**
 * \ingroup prob_dists
 * \defgroup multivar_dists Multivariate Distributions
 * Distributions with Matrix inputs
 */
/**
 * \ingroup prob_dists
 * \defgroup univar_dists Univariate Distributions
 * Distributions with scalar, vector, or array input.
 */

#include <stan/math/rev.hpp>

#include <stan/math/torsten/torsten.hpp>
using namespace torsten;
#endif
\end{minted}

One can generate C++ code documention for Torsten using the same
doxygen process as in \texttt{stan/math}.
\begin{minted}[breaklines=true,fontsize=\footnotesize,breakanywhere=true]{bash}
make doxygen
\end{minted}
To access the generated Torsten documentation, point the browser to
\begin{minted}[breaklines=true,fontsize=\footnotesize,breakanywhere=true]{bash}
/stan_math/doc/api/html/index.html
\end{minted}
and find \texttt{torsten} namespace.

\chapter{\texttt{Stan}}
\label{sec:orgf458996}
The forked \texttt{stan} serves passing \texttt{stan/math} through as well as
testing ground for experimental inference algorithms such as
cross-chain warmup (see Section 6.2).

\chapter{\texttt{CmdStan}}
\label{sec:org2a9133b}
Similar to \texttt{Stan}, the forked command line interface has boilerplate
code for the cross-chain warmup algorithm. It also contains
Torsten-specific \texttt{makefile} flags. Similar to its upstream repo,
the making process will download a \texttt{stanc3} binary in order to
transpile \texttt{Stan} code to \texttt{C++} code. Changes have been made in the
repo to download Torsten-compatible \texttt{stanc3} binary so that the
transpiler recoganizes Torsten function signatures.

\chapter{MPI parallelization}
\label{sec:orgdcb8853}
As an alternative to Stan's \texttt{reduce\_sum} function designed for
multicore infrastrue, Torsten provides MPI parallelization for
population models as well as experimental cross-chain warmup model.

\section{Population solver}
\label{sec:org2f84dde}
The population solver functions \texttt{pmx\_solve\_group\_rk45|bdf|adams} have similar
construct to their single-subject counterparts. For example
\begin{minted}[breaklines=true,fontsize=\footnotesize,breakanywhere=true]{cpp}
template <typename F, typename... Ts>
auto pmx_solve_group_bdf(const F& f, const int nCmt,
                         const std::vector<int>& len, Ts... args) {
  return PMXSolveGroupODE<dsolve::PMXOdeIntegrator<dsolve::PMXVariadicOdeSystem, dsolve::PMXCvodesIntegrator<CV_BDF, CV_STAGGERED>>>::solve(f, nCmt, len, args...);
}
\end{minted}
is the group solver version of \texttt{pmx\_solve\_bdf}. The only difference is
instead of using \texttt{PMXSolveODE} here we use \texttt{PMXSolveGroupODE} class to
numerically solve the \emph{population's} ODEs.

When looping through events, group solver distributes the population
to parallel processes for ODE solution. At the end of each event, the
solver collects results of the entire population from the
processes. This mechanism is implemented in the \texttt{EventSolver} class,
along with its sequential version.

\section{Cross-chain warmup}
\label{sec:orga18906d}
The experimental algorithm sits on top of Stan's sampler engine.
\begin{minted}[breaklines=true,fontsize=\footnotesize,breakanywhere=true]{bash}
Torsten/cmdstan/src/stan/mcmc/cross_chain
\end{minted}
The sampler is modified to include the cross-chain adaptation.
For example, function \texttt{adapt\_diag\_e\_nuts::transition} becomes
\begin{minted}[breaklines=true,fontsize=\footnotesize,breakanywhere=true]{cpp}
sample transition(sample& init_sample, callbacks::logger& logger) {
  sample s = diag_e_nuts<Model, BaseRNG>::transition(init_sample, logger);

  if (this->adapt_flag_) {
    this->stepsize_adaptation_.learn_stepsize(this->nom_epsilon_,
                                              s.accept_stat());

    if (this -> use_cross_chain_adapt()) {
      /// cross chain adapter has its own var adaptor so needs to add sample
      this -> add_cross_chain_sample(s.log_prob(), this -> z().q);
      bool update = this -> cross_chain_adaptation(this -> z().inv_e_metric_, logger);
      if (update) {
        // this->init_stepsize(logger);
        double new_stepsize = this -> cross_chain_stepsize(this->nom_epsilon_);
        this -> set_nominal_stepsize(new_stepsize);
        this->stepsize_adaptation_.set_mu(log(10 * this->nom_epsilon_));
        this->stepsize_adaptation_.restart();
      }
    } else {
      bool update = this->var_adaptation_.learn_variance(this->z_.inv_e_metric_,
                                                         this->z_.q);
      if (update) {
        this->init_stepsize(logger);
        this->stepsize_adaptation_.set_mu(log(10 * this->nom_epsilon_));
        this->stepsize_adaptation_.restart();
      }
    }
  }
  return s;
}
\end{minted}
The \texttt{this -> use\_cross\_chain\_adapt} condition controls if cross-chain
warmup is used and choose the adaptation accordingly.

\chapter{\texttt{stanc3}}
\label{sec:org4ca4b46}
The forked \texttt{stanc3} contains a
\begin{minted}[breaklines=true,fontsize=\footnotesize,breakanywhere=true]{bash}
stanc3/src/middle/torsten.ml
\end{minted}
file for Torsten function signatures. Unlike Stan functions, functions
like \texttt{pmx\_solve\_rk45} supports a long list of signatures in order to
allow a combination of
\begin{itemize}
\item parameters shared by the entire population vs subject-specifc,
\item time-independent parameters vs time-dependent parameters,
\item default (thus omittable) \(F=1.0\) vs user-specified bioavailability,
\item default (thus omittable) \(t_{\text{Lag}}=0.0\) vs user-specified lag time,
\item default (thus omittable) vs user-specified control parameters.
\end{itemize}

Any higher-order Torsten function must have its signature defined in
\texttt{torsten.ml} in order to be recognized by the transpiler. Thus the
workflow of adding a Torsten function usually is to first implement
the function in \texttt{torsten\_stan} followed by adding its signature in \texttt{torsten.ml}.

\section{Adding a new function}
\label{sec:orge713257}
Now let us walkthrough how a new function is added in Torsten using
\texttt{pmx\_solve\_linode} function as example. The function solves
dosing events using a linear ODE model, specified as a coefficient
matrix for the RHS of a linear ODE.

First we implement the C++ function in \texttt{torsten\_math}. As one can find
at
\begin{minted}[breaklines=true,fontsize=\footnotesize,breakanywhere=true]{bash}
stan_math/stan/math/torsten/pmx_solve_linode.hpp
\end{minted}
the implementation should be in \texttt{torsten} namespace.

\begin{minted}[breaklines=true,fontsize=\footnotesize,breakanywhere=true]{c++}
namespace torsten {
  // ...
template <typename T0, typename T1, typename T2, typename T3,
          typename T4, typename T5, typename T6>
stan::matrix_return_t<T0, T1, T2, T3, T4, T5, T6>
pmx_solve_linode(const std::vector<T0>& time,
                 const std::vector<T1>& amt,
                 const std::vector<T2>& rate,
                 const std::vector<T3>& ii,
                 const std::vector<int>& evid,
                 const std::vector<int>& cmt,
                 const std::vector<int>& addl,
                 const std::vector<int>& ss,
                 const std::vector< Eigen::Matrix<T4, -1, -1> >& system,
                 const std::vector<std::vector<T5> >& biovar,
                 const std::vector<std::vector<T6> >& tlag) {
  // ...
}
}
\end{minted}

As part of test-based development process, there multiple unit tests for this function
\begin{minted}[breaklines=true,fontsize=\footnotesize,breakanywhere=true]{c++}
stan_math/stan/math/torsten/test/unit/linode_typed_finite_diff_test.cpp
stan_math/stan/math/torsten/test/unit/linode_typed_overload_test.cpp
stan_math/stan/math/torsten/test/unit/linode_typed_test.cpp
\end{minted}
for testing with finite-difference results, overloaded function
signature, and solution correctness, respectively.

To have the Stan language recoganize a regular function like
\texttt{pmx\_solve\_linode} we only need to add its signature to the
aforementioned \texttt{torsten.ml} file. For the above signature, one can
find the corresponding signature
\begin{minted}[breaklines=true,fontsize=\footnotesize,breakanywhere=true]{ocaml}
add_func
  ( "pmx_solve_linode"
  , ReturnType UMatrix
  , [ (AutoDiffable, UArray UReal)       (* time *)
    ; (AutoDiffable, UArray UReal)       (* amt *)
    ; (AutoDiffable, UArray UReal)       (* rate *)
    ; (AutoDiffable, UArray UReal)       (* ii *)
    ; (DataOnly, UArray UInt)            (* evid *)
    ; (DataOnly, UArray UInt)            (* cmt *)
    ; (DataOnly, UArray UInt)            (* addl *)
    ; (DataOnly, UArray UInt)            (* ss *)
    ; (AutoDiffable, UArray UMatrix)     (* pMatrix *)
    ; (AutoDiffable, (UArray (UArray UReal)))     (* biovar *)
    ; (AutoDiffable, (UArray (UArray UReal))) ],  (* tlag *)
  Common.Helpers.AoS) ;
\end{minted}
One can easily map the above argumentss and return value to the C++
function. Note that we need to make the Stan language aware of whether
an argument could be parameter (\texttt{AutoDiffable}) or not (\texttt{DataOnly}).

\section{Adding a new high-order function}
\label{sec:org532ee4b}
Torsten's numerical ODE solvers are high-order functions, i.e. functions with function arguments.
Adding a new high-order function is slightly more complicated.
Let us use \texttt{pmx\_solve\_rk45} as an example. As shown in
Section 1, the function uses variadic arguments in order to support
different signatures. Here let us assume the one of them as follows.
\begin{minted}[breaklines=true,fontsize=\footnotesize,breakanywhere=true]{cpp}
namespace torsten {
    template <typename T0, typename T1, typename T2, typename T3, typename T4,
              typename T5, typename T6, typename F>
    static stan::matrix_return_t<T0, T1, T2, T3, T4, T5, T6>
    pmx_solve_rk45(const F& f,
                   const int nCmt,
                   const std::vector<T1>& amt,
                   const std::vector<T2>& rate,
                   const std::vector<T3>& ii,
                   const std::vector<int>& evid,
                   const std::vector<int>& cmt,
                   const std::vector<int>& addl,
                   const std::vector<int>& ss,
                   const std::vector<std::vector<T4> >& pMatrix,
                   const std::vector<std::vector<T5> >& biovar,
                   const std::vector<std::vector<T6> >& tlag,
                   std::ostream* msgs) {//...}
}
\end{minted}
Note that now we have \(f\) for the ODE RHS, \texttt{nCmt} as number of
compartments, and \texttt{msgs} for I/O of the ODE solver messages.

In order to support the above signature in Stan, in the \texttt{torsten.ml}
we need first define the ODE function signature.
\begin{minted}[breaklines=true,fontsize=\footnotesize,breakanywhere=true]{ocaml}
let pmx_solve_ode_func =
  [ ( UnsizedType.AutoDiffable
    , UnsizedType.UFun
        ( [ (UnsizedType.AutoDiffable, UnsizedType.UReal) (* time *)
          ; (UnsizedType.AutoDiffable, UnsizedType.UVector) (* states *)
          ; (UnsizedType.AutoDiffable, UArray UReal) (* real param *)
          ; (DataOnly, UArray UReal); (DataOnly, UArray UInt) ] (* int param *)
        , ReturnType UnsizedType.UVector (* return type *)
        , FnPlain, AoS) ) ]
\end{minted}

Now we can add the signature as \footnote{The actual code in \texttt{torsten.ml} is more complicated in order to
accomondate many variants of the above signature.}
\begin{minted}[breaklines=true,fontsize=\footnotesize,breakanywhere=true]{ocaml}
add_func
  ( "pmx_solve_linode"
  , ReturnType UMatrix
  , [ pmx_solve_ode_func                         (* f *)
    ; (UnsizedType.DataOnly, UnsizedType.UInt)   (* nCmt *)
    ; (AutoDiffable, UArray UReal)       (* time *)
    ; (AutoDiffable, UArray UReal)       (* amt *)
    ; (AutoDiffable, UArray UReal)       (* rate *)
    ; (AutoDiffable, UArray UReal)       (* ii *)
    ; (DataOnly, UArray UInt)            (* evid *)
    ; (DataOnly, UArray UInt)            (* cmt *)
    ; (DataOnly, UArray UInt)            (* addl *)
    ; (DataOnly, UArray UInt)            (* ss *)
    ; (AutoDiffable, UArray UMatrix)     (* pMatrix *)
    ; (AutoDiffable, (UArray (UArray UReal)))     (* biovar *)
    ; (AutoDiffable, (UArray (UArray UReal))) ],  (* tlag *)
  Common.Helpers.AoS) ;
\end{minted}

\chapter{\texttt{Torsten} container}
\label{sec:orgd59585c}
All the above repos are collected in the container repo
\texttt{Torsten} using \texttt{git subtree} command, so that user only needs to
clone this repo in order to use Torsten. Documentation and example
models can also be found in this repo. Note that the development of
Torsten functionalilties still happens in each submodule.

In the rest of the section we describe the process to upate \texttt{Torsten}
to certain Stan release, using Stan's \texttt{master} branch as example.

\section{\texttt{Math}}
\label{sec:org8bcb265}
Assume at Torsten's \texttt{Math} repo there are remote
\begin{minted}[breaklines=true,fontsize=\footnotesize,breakanywhere=true]{bash}
bash-3.2$ git remote -vv
origin    git@github.com:metrumresearchgroup/math.git (fetch)
origin    git@github.com:metrumresearchgroup/math.git (push)
stan-dev  https://github.com/stan-dev/math.git (fetch)
stan-dev  https://github.com/stan-dev/math.git (push)
\end{minted}
and local branches
\begin{minted}[breaklines=true,fontsize=\footnotesize,breakanywhere=true]{bash}
bash-3.2$ git branch
  develop                       # upstream stan/math develop
  master                        # upstream stan/math master
  torsten-develop               # torsten develop
  torsten-master                # torsten master
\end{minted}
so that we can update \texttt{torsten-develop}
\begin{minted}[breaklines=true,fontsize=\footnotesize,breakanywhere=true]{bash}
bash-3.2$ git branch
  git merge -Xtheirs master     # may need to manually resolve conflicts
\end{minted}
Before pushing, make sure the Torsten unit tests are passed. If not,
one may need to iterate between \texttt{tortsen\_math} and \texttt{math} repo to
ensure the implementations are consistent.
\begin{minted}[breaklines=true,fontsize=\footnotesize,breakanywhere=true]{bash}
bash-3.2$ make clean-all; ./runTests.py -j4 stan/math/torsten/test/unit/
\end{minted}
One should run unit tests twice, with and without \texttt{TORSTEN\_MPI} in
\texttt{make/local}, for sequential and parallel tests, respectively \footnote{To run parallel jobs one needs a properly installed MPI
library. Setting \texttt{TORSTEN\_MPI=1} in the \texttt{make/local} prompts
comopiling the tests using \texttt{mpicxx}, the MPI-enabled C++ compiler. One
can also specify the MPI comopiler path in \texttt{make/local} by setting
\texttt{CXX} and \texttt{CC} variables.}.
Currently there is a GitHub Actions workflow for sequential unit tests
on Windows, Ubuntu, and MacOS platforms. The action is performed
automatically when one push to Torsten's \texttt{math} repo fork.

\section{\texttt{Stan}}
\label{sec:orgaac0db9}
The treatment for \texttt{Stan} repo follows a same procedure.

\section{\texttt{stanc3}}
\label{sec:orgb6072a8}
Any added/update Torsten function should have its signature registered
in \texttt{stanc3} repo. Similar to the previous two repos, we need to first
merge from upstream.
\begin{minted}[breaklines=true,fontsize=\footnotesize,breakanywhere=true]{bash}
bash-3.2$ git branch
  master                        # stanc3 upstream
  torsten-develop               # torsten branch
bash-3.2$ git checkout master && git pull
bash-3.2$ git checkout torsten-develop && git merge master
\end{minted}
Before merge one should use models in \texttt{test/integration/good/torsten}
for integration tests (TODO: need to add these models to \texttt{dune} tests).

Before updating \texttt{cmdstan}, we need to create a \texttt{stanc3} binary for
\texttt{make} script to download. This is done using \href{https://github.com/metrumresearchgroup/stanc3/actions}{the \texttt{Build binaries}
action}. After finish one can collect the binaries as artifacts and
uploaded them to a new release. (TODO: this process can be automated).


\section{\texttt{cmdstan}}
\label{sec:orgb3dd27a}
Updating \texttt{cmdstan} repo requires the same procedure as \texttt{math} and
\texttt{stan}. In addition, one must update the \texttt{TORSTEN\_STANC3\_VERSION}
variable in \texttt{make/torsten\_stanc.mk} to reflect the \texttt{stanc3}
release. For example, if the latest \texttt{stanc3} release for Torsten is
\texttt{torsten\_v0.90.0rc2}, then one must set
\begin{minted}[breaklines=true,fontsize=\footnotesize,breakanywhere=true]{bash}
TORSTEN_STANC3_VERSION=torsten_v0.90.0rc2
\end{minted}
so that correct \texttt{stanc3} binaries will be downloaded. This is done by
the \texttt{make/stanc} script, and one can check if the link there
\begin{minted}[breaklines=true,fontsize=\footnotesize,breakanywhere=true]{bash}
https://github.com/metrumresearchgroup/stanc3/releases/download/$(TORSTEN_STANC3_VERSION)/$(OS_TAG)-stanc
\end{minted}
correctly points to binary artifacts geneated in the previous
section. In the above link \texttt{OS\_TAG} equals to \texttt{mac}, \texttt{windows}, or
\texttt{ubuntu}.

\section{\texttt{Torsten}}
\label{sec:org63cb8c1}
The container repo is updated after all the above modules are in
place. We need first create a \texttt{remote} for each of the above modules
\begin{minted}[breaklines=true,fontsize=\footnotesize,breakanywhere=true]{bash}
bash-3.2$ git remote -vv
origin  git@github.com:metrumresearchgroup/Torsten.git (fetch)
origin  git@github.com:metrumresearchgroup/Torsten.git (push)
cmdstan git@github.com:metrumresearchgroup/cmdstan.git (fetch)
cmdstan git@github.com:metrumresearchgroup/cmdstan.git (push)
math    git@github.com:metrumresearchgroup/math.git (fetch)
math    git@github.com:metrumresearchgroup/math.git (push)
stan    git@github.com:metrumresearchgroup/stan.git (fetch)
stan    git@github.com:metrumresearchgroup/stan.git (push)
stanc3  git@github.com:metrumresearchgroup/stanc3.git (fetch)
stanc3  git@github.com:metrumresearchgroup/stanc3.git (push)
torsten_math    git@github.com:metrumresearchgroup/torsten_math.git (fetch)
torsten_math    git@github.com:metrumresearchgroup/torsten_math.git (push)
\end{minted}
so that next we can call
\begin{minted}[breaklines=true,fontsize=\footnotesize,breakanywhere=true]{bash}
./substree_update.sh cmdstan_branch stan_branch math_branch torsten_math_branch
\end{minted}
to create the container repo based on the branch name given for each
module. By default the script uses \texttt{torsten-develop} branch for
\texttt{cmdstan}, \texttt{stan}, \texttt{math}, and \texttt{develop} branch for
\texttt{torsten\_math}. That is,
\begin{minted}[breaklines=true,fontsize=\footnotesize,breakanywhere=true]{bash}
./substree_update.sh
\end{minted}
is equivalent to
\begin{minted}[breaklines=true,fontsize=\footnotesize,breakanywhere=true]{bash}
./substree_update.sh torsten-develop torsten-develop torsten-develop develop
\end{minted}

After updating the container we can revise the documentation and
example models in the repo accordingly.
\end{document}
